\documentclass[UTF8]{ctexart}

% --- 导言区 (Preamble) ---
% 1. 基础数学与排版
\usepackage{amsmath}
\usepackage{amssymb}
\usepackage{geometry}
\geometry{a4paper, margin=1in}

% 2. 绘图与图片控制
\usepackage{graphicx}
\usepackage{float}      % 【新增】用于强制固定图片位置 [H]
\usepackage{tikz}       % 绘图工具
\usetikzlibrary{patterns, calc} % 引入calc库以支持更复杂的坐标计算

% 3. 文档信息
\title{微积分随堂练习}
\author{pvzercoffee}
\date{\today}

\begin{document}

\maketitle

% --- 第一部分:积分 ---
\section{题目:计算不定积分}

求下列不定积分:
\begin{equation}
    I = \int \frac{1}{1 + \sin x} \,dx
\end{equation}

\subsection*{解:}利用 万能代换公式(Universal Substitution),设 $t = \tan \frac{x}{2}$,则有:
\begin{itemize}
    \item $dx = \frac{2}{1 + t^2} dt$
    \item $\sin x = \frac{2t}{1 + t^2}$
\end{itemize}

将上述代换代入原式:
\begin{align*}
    I &= \int \frac{1}{1 + \frac{2t}{1+t^2}} \cdot \frac{2}{1+t^2} \,dt \\
      &= \int \frac{2}{1 + t^2 + 2t} \,dt \\
      &= \int \frac{2}{(1 + t)^2} \,dt
\end{align*}

最终计算得出:
\begin{equation}
    I = -\frac{2}{1 + t} + C = -\frac{2}{1 + \tan \frac{x}{2}} + C
\end{equation}

\newpage

% --- 第二部分:应用题 ---
\section{应用题:防空洞截面优化}

% 使用 enumerate 环境自动编号,比手写 "12." 更优雅
\begin{enumerate}
    \item[12.] 某地区防空洞的截面拟建成矩形加半圆(如图 3-19). 截面的面积为 $5 \, \mathrm{m}^2$. 问底宽 $x$ 为多少时才能使截面的周长最小,从而使建造时所用的材料最省?
\end{enumerate}

% 使用 [H] 强制让图片呆在这里,不要乱跑
\begin{figure}[H]
    \centering
    \begin{tikzpicture}[scale=1.5]
        % 【建议】使用 pgfmathsetmacro 进行数值定义,比 def 更适合计算
        \pgfmathsetmacro{\w}{2}   % 宽度 x
        \pgfmathsetmacro{\h}{1.5} % 高度 y
        
        % 1. 画矩形的下半部分(左-下-右)
        \draw[thick] (0,\h) -- (0,0) -- (\w,0) -- (\w,\h);
        
        % 2. 画顶部的半圆
        % arc 语法: (起点) arc (起始角度:终止角度:半径)
        \draw[thick] (\w,\h) arc (0:180:{\w/2});
        
        % 3. 画中间的虚线
        \draw[dashed] (0,\h) -- (\w,\h);
        
        % 4. 标注尺寸 x 和 y
        \draw (0,0) -- node[below] {$x$} (\w,0);
        \draw (\w,0) -- node[right] {$y$} (\w,\h);
        
        % 5. 添加图注 (利用 calc 库计算中点更稳健,或者直接算好坐标)
        \node at (\w/2, -0.5) {\small 图 3-19};
    \end{tikzpicture}
\end{figure}

\subsection*{解:}

列出周长与面积:

\begin{align*}
    &L=2y+x+ \frac{1}{2} \pi x \\
    &S=xy+ \frac{ \pi }{8}=5 \\
\end{align*}
通过$S$找到$y$的$x$表达式:
\begin{equation}
    y = \frac{5}{x}- \frac{\pi}{8} x^{2}
\end{equation}
将表达式代入周长$L$,求出周长的$x$函数:
\begin{equation}
    L= \frac{10}{x} + \frac{\pi}{4} x+x
\end{equation}
对$L$求导一阶导、二阶导
\begin{align*}
    &L'= 1+ \frac{\pi}{4} - \frac{10}{x^{2}} \\
    &L'' = \frac{20}{x^{3}}
\end{align*}
令$L'=0$,得到$x=\sqrt{\frac{40}{4+\pi}}$(唯一驻点);

将其代入$L''$,得$L''>0$,即$x=\sqrt{\frac{40}{4+\pi}}$为$L$的极小值点、最小值点。
\\

故,当底宽$x$为$\sqrt{\frac{40}{4+\pi}}$时,截面的周长最小。
\newpage


\section{题目:含参函数的零点个数}

\begin{enumerate}
    \item[] 设常数 $k > 0$,函数 $f(x) = \ln x - \frac{x}{e} + k$ 在 $(0, +\infty)$ 内零点的个数为 \underline{\hspace{4cm}} .
\end{enumerate}

\subsection*{解:}
$f(x)$的定义域为$\left( 0, + \infty \right)$,求得
\begin{equation*}
    f'(x)= \frac{1}{x}-\frac{1}{e} 
\end{equation*}
令$f'(x)=0$,得$x=e$为唯一驻点。

\begin{align*}
    x \in (0,e),&f'(x) > 0 \\
    x = e,&f'(x) = 0 \\
    x \in (e,+\infty),&f'(x) < 0 \\
\end{align*}

由此可知,$x=e$为$f(x)$的唯一极大值点,即最大值点,且
\begin{align*}
    &\lim_{x \to 0^{+}} f(x) = - \infty \\
    &\lim_{x \to +\infty}f(x) = -\infty
\end{align*}

又因为极大值$k>0$,因此,函数$f(x)$在$(0,+\infty)$的零点个数为2个。

\end{document}